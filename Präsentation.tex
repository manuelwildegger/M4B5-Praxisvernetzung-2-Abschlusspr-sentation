\documentclass{Präsentationsvorlage}
%Einstellungen Notizen
%\setbeameroption{hide notes}
%\setbeameroption{show notes on second screen=bottom}

%Titelseite
%Titel
\title{\LARGE{Abschlusskolloquium}}
\subtitle{Selbsteinschätzung intensivpflegerischer Fachkompetenzen im Längsschnitt}
\date{23. Januar 2026}
\author{Manuel Wildegger}
\institute{Modul M4B5: Praxisvernetzung 2\\Prof. Dr. Kerres \& Prof. Dr. Mittenzwei\\Katholische Stiftungshochschule München}

\begin{document}

\begin{frame}[plain]
    \maketitle
\end{frame}

\begin{frame}[plain]
    \tableofcontents[pausesections]
\end{frame}

\section{Forschungsfrage}
\begin{frame}{Forschungsfrage}
\begin{center}
    Wie verändert sich die Selbsteinschätzung der im F!t for ICU-Modul \textit{Menschen mit Herzkreislauferkrankungen pflegen} erlernten Fachkompetenzen bei Teilnehmern mit generalistischer Ausbildung über die Dauer der Teilnahme?
    \end{center}
\end{frame}

\section{Datenerhebung}
\subsection{Datenerhebungsinstrument}
\begin{frame}{Anpassung des Fragebogens}
\begin{block}{Finale Fragebogenstruktur}
    \begin{enumerate}
        \item Einverständnis mit Dropout
        \item Persönlicher Code
        \item Kompetenzen des Teilmoduls 3.4.
        \begin{enumerate}
            \item Kompetenz 1 mit 6 Items
            \item Kompetenz 2 mit 5 Items
            \item Kompetenz 3 mit 5 Items
            \item Kompetenz 4 mit 7 Items
        \end{enumerate}
        \item Berufsabschluss
    \end{enumerate}
\end{block}
\end{frame}
\subsection{Datenerhebung}
\begin{frame}{Datenerhebung}
    \begin{itemize}
        \item t1: 17. November 2025
        \item t2: 18. Dezember 2025
        \item Befragungszeitpunkt jeweils zu Beginn des Unterrichts (8:30 Uhr)
        \item Bereitstellung per qr-Code
    \end{itemize}
\end{frame}

\section{Datenauswertung}
\begin{frame}{Methodik}
\begin{itemize}
    \item R-Skript
    \item Deskriptive Statistik von t1 und t2
    \item Grundsatzentscheidung: Likert-Skala = Ordinal
    \item Nicht-parametrische Testung, 
\end{itemize}
\end{frame}

\begin{frame}{Bereinigung des Datensatzes (global)}

\end{frame}

\begin{frame}[plain]
\begin{figure}
    \centering
    \includegraphics[width=\linewidth]{Ausbildung.png}
    \caption{Studienteilnehmende nach Berufsabschluss und Befragungszeitpunkt geteilt}
\end{figure}
\end{frame}

\begin{frame}[plain]
\begin{figure}
    \centering
    \includegraphics[width=\linewidth]{aw_k1.png}
    \caption{Deskriptive Statistik Kompetenz 1}
\end{figure}
\end{frame}

\begin{frame}[plain]
\begin{figure}
    \centering
    \includegraphics[width=\linewidth]{aw_k2.png}
    \caption{Deskriptive Statistik Kompetenz 2}
\end{figure}
\end{frame}

\begin{frame}[plain]
\begin{figure}
    \centering
    \includegraphics[width=\linewidth]{aw_k3.png}
    \caption{Deskriptive Statistik Kompetenz 3}
\end{figure}
\end{frame}

\begin{frame}[plain]
\begin{figure}
    \centering
    \includegraphics[width=\linewidth]{aw_k4.png}
    \caption{Deskriptive Statistik Kompetenz 4}
\end{figure}
\end{frame}

\begin{frame}[allowframebreaks]{Literatur}
\printbibliography
\end{frame}

\end{document}
